\documentclass[a4paper]{article}
\usepackage[utf8x]{inputenc}
\usepackage[portuguese]{babel}
\usepackage{graphicx}
\usepackage{a4wide}
\usepackage[pdftex,hidelinks]{hyperref}
\usepackage{float}
\usepackage{indentfirst}
\usepackage{subcaption}
\usepackage[cache=false]{minted}
\usepackage{amsmath}
\usepackage{listings}
\usepackage{color}

\definecolor{dkgreen}{rgb}{0,0.6,0}
\definecolor{gray}{rgb}{0.5,0.5,0.5}
\definecolor{mauve}{rgb}{0.58,0,0.82}

\lstset{frame=tb,
language=C++,
aboveskip=3mm,
belowskip=3mm,
showstringspaces=false,
columns=flexible,
basicstyle={\small\ttfamily},
numbers=none,
numberstyle=\tiny\color{gray},
keywordstyle=\color{blue},
commentstyle=\color{dkgreen},
stringstyle=\color{mauve},
breaklines=true,
breakatwhitespace=true,
tabsize=4
}

\newcommand{\x}{\times}

\begin{document}

\title{Computação Gráfica\\ Transformações Geométricas}
\author{Bárbara Cardoso (a80453) \and Márcio Sousa (a82400) \and Pedro Mendes (a79003)}
\date{\today}

\begin{titlepage}

    %título
    \thispagestyle{empty}
    \begin{center}
        \begin{minipage}{0.75\linewidth}
            \centering
            %engenharia logo
            \includegraphics[width=0.4\textwidth]{eng.jpeg}\par\vspace{1cm}
            \vspace{1.5cm}
            %títulos
            \href{https://www.uminho.pt/PT}{\scshape\LARGE Universidade do Minho} \par
            \vspace{1cm}
            \href{https://www.di.uminho.pt/}{\scshape\Large Departamento de Informática} \par
            \vspace{1.5cm}

            \maketitle
        \end{minipage}
    \end{center}

\end{titlepage}

\tableofcontents

\pagebreak

\section{Introdução}

\section{Estrutura dos ficheiros usados pelo programa}

Os ficheiros utilizados pelo programa são definidos em formato \texttt{xml} com as seguintes tags:
\begin{itemize}
    \item \texttt{<scene>} Define o inicio de uma cena. Serve o papel de um
        super grupo e não tem atributos
    \item \texttt{<group>} Define um grupo de transformações, modelos e
        subgrupos, e aceita como atributos as cores com que devem ser pintados
        os modelos, por exemplo,
        \texttt{<group R='1'> \ldots </group>} pinta os modelos de vermelho.
        Para além disso os subgrupos deste irão herdar estas cores a não ser
        que definam as suas próprias cores. Os atributos que podem ser usados
        para isto sao R, G, B e A, que correspondem a \textit{Red},
        \textit{Green}, \textit{Blue} e \textit{Alpha}.
    \item \texttt{<models>} e \texttt{<model>} Definem os modelos a desenhar
        num grupo. O primeiro contem uma lista do segundo e este tem o atributo
        \texttt{file} que indica o ficheiro \texttt{.3d} a usar.
    \item \texttt{<translate>}, \texttt{<rotate>} e \texttt{<scale>} Definem as
        transformações geométricas que podem ser usadas para transformar os
        modelos e subgrupos. Atributos de cada um destes:
        \begin{itemize}
            \item \textbf{Translate:} X, Y, Z.
            \item \textbf{Rotate:} angle, X, Y, Z
            \item \textbf{Scale:} X, Y, Z
        \end{itemize}
\end{itemize}

Exemplo de uma ficheiro de input:
\input{example_xml.tex}

\section{Arquitetura do Projecto}

\section{Gestão de memória}

\section{Conclusões e Trabalho Futuro}

\end{document}

