\documentclass[a4paper]{article}
\usepackage[utf8x]{inputenc}
\usepackage[portuguese]{babel}
\usepackage{graphicx}
\usepackage{a4wide}
\usepackage[pdftex,hidelinks]{hyperref}
\usepackage{float}
\usepackage{indentfirst}
\usepackage{subcaption}
\usepackage[cache=false]{minted}
\usepackage{amsmath}
\usepackage{listings}
\usepackage{color}

\definecolor{dkgreen}{rgb}{0,0.6,0}
\definecolor{gray}{rgb}{0.5,0.5,0.5}
\definecolor{mauve}{rgb}{0.58,0,0.82}

\lstset{frame=tb,
language=C++,
aboveskip=3mm,
belowskip=3mm,
showstringspaces=false,
columns=flexible,
basicstyle={\small\ttfamily},
numbers=none,
numberstyle=\tiny\color{gray},
keywordstyle=\color{blue},
commentstyle=\color{dkgreen},
stringstyle=\color{mauve},
breaklines=true,
breakatwhitespace=true,
tabsize=4
}

\newcommand{\x}{\times}

\begin{document}

\title{Computação Gráfica\\ Transformações Geométricas}
\author{Bárbara Cardoso (a80453) \and Márcio Sousa (a82400) \and Pedro Mendes (a79003)}
\date{\today}

\begin{titlepage}

    %título
    \thispagestyle{empty}
    \begin{center}
        \begin{minipage}{0.75\linewidth}
            \centering
            %engenharia logo
            \includegraphics[width=0.4\textwidth]{eng.jpeg}\par\vspace{1cm}
            \vspace{1.5cm}
            %títulos
            \href{https://www.uminho.pt/PT}{\scshape\LARGE Universidade do Minho} \par
            \vspace{1cm}
            \href{https://www.di.uminho.pt/}{\scshape\Large Departamento de Informática} \par
            \vspace{1.5cm}

            \maketitle
        \end{minipage}
    \end{center}

\end{titlepage}

\tableofcontents

\pagebreak

\section{Introdução}

\section{Estrutura dos ficheiros usados pelo programa}\label{sec:estrutura-ficheiros}

Os ficheiros utilizados pelo programa são definidos em formato \texttt{xml} com as seguintes tags:
\begin{itemize}
    \item \texttt{<scene>} Define o inicio de uma cena. Serve o papel de um
        super grupo e não tem atributos
    \item \texttt{<group>} Define um grupo de transformações, modelos e
        subgrupos, e aceita como atributos as cores com que devem ser pintados
        os modelos, por exemplo,
        \texttt{<group R='1'> \ldots </group>} pinta os modelos de vermelho.
        Para além disso os subgrupos deste irão herdar estas cores a não ser
        que definam as suas próprias cores. Os atributos que podem ser usados
        para isto sao R, G, B e A, que correspondem a \textit{Red},
        \textit{Green}, \textit{Blue} e \textit{Alpha}. Alternativamente, pode ser passado
        o parametro \textit{RAND} para que a cor de cada triangulo ser
        escolhida aleatoriamente.
    \item \texttt{<models>} e \texttt{<model>} Definem os modelos a desenhar
        num grupo. O primeiro contem uma lista do segundo e este tem o atributo
        \texttt{file} que indica o ficheiro \texttt{.3d} a usar.
    \item \texttt{<translate>}, \texttt{<rotate>} e \texttt{<scale>} Definem as
        transformações geométricas que podem ser usadas para transformar os
        modelos e subgrupos. Atributos de cada um destes:
        \begin{itemize}
            \item \textbf{Translate:} X, Y, Z.
            \item \textbf{Rotate:} angle, X, Y, Z
            \item \textbf{Scale:} X, Y, Z
        \end{itemize}
\end{itemize}

Exemplo de uma ficheiro de input:
\input{example_xml.tex}

\pagebreak

\section{Arquitetura do Projecto}

\subsection{Camera}

A camera não sofreu grandes alterações desde a fase anterior, foi apenas movida
para um \textit{namespace} à parte.

\subsection{Model}

O Model não sofreu grandes alterações, para além de poder ser desenhado com cores aleatórias ou
usando cores aleatórias ou sem cores (``herdando'' a ultima cor que tenha sido
seleccionada).

\subsection{Render}

O render foi criado para mover a lógica de desenho para um \textit{namespace} à
parte, e alterado para acomodar o resto das modificações.

\subsection{Group}

O \texttt{Group} é a classe principal desta fase do projecto, esta foi
desenhada para imitar o formato do xml, possuindo então uma sequência de
transformações, um conjunto de modelos e por fim os seus subgrupos, ficando
assim com uma definição recursiva. Este é construído directamente a partir do
xml, e disponibiliza uma função de desenho que pede o a profundidade da arvore
de grupos e subgrupos que se pretende desenhar. Esta função funciona da
seguinte forma:

Caso o numero passado seja maior do que 0, então começa por aplicar todas as
suas transformações, de seguida desenha todos os modelos, neste passo tem em
atenção de aplicar as cores que tenha, ou por passagem directa ou por herança,
ou então de desenhar o modelo com cores aleatórias
(ver~\ref{sec:estrutura-ficheiros}). Por fim, chama recursivamente o método de
desenho dos subgrupos passando menos um do que a profundidade que recebeu como
parâmetro.

\subsection{Transformations}

As transformações foram definidas como uma interface que obriga a
implementação do método \texttt{transform}. Foram definidas também 3 implementações
da mesma, \texttt{Rotate}, \texttt{Translate} e \texttt{Scale}, correspondentes
às 3 transformações possíveis.

\section{Novas Primitivas}

\subsection{Torus}

\begin{figure}[H]
    \includegraphics[width=\textwidth]{torus.png}
    \caption{Torus: \textit{innerRadius}: 1, \textit{outerRadius}: 3, \textit{sides}: 10, \textit{rings}: 20}
\end{figure}

Um torus é definido por duas circunferências, uma interior e outra exterior, em
que a exterior tem um raio de $outerRadius$ e a circunferência interior um raio
de $outerRadius - 2 \x innerRadius$.
O $innerRadius$ é o raio de cada anel situado entre as duas circunferências. Os
centros destes anéis estão distanciados segundo um ângulo de $\alpha_{step}$,
usando coordenadas polares.

\begin{figure}[H]
    \centering
    % \includegraphics{}
    \caption{Torus completo}
\end{figure}

Para cada par de anéis, são desenhados uma série de rectângulos a ligar um anel para cada para fazer a ``parede'' do torus. Estes têm como vértices os pontos $p_0$, $p_1$, $p_2$ e $p_3$ que são calculados por coordenadas polares, que têm como centro de aplicação o centro do anel a que pertencem (ponto $B$). O espaçamento destes pontos segue, por sua vez, um ângulo de $\beta_{step}$.

\begin{figure}[H]
    \centering
    % \includegraphics{}
    \caption{Corte do torus com um par de anéis}
\end{figure}

\[B =  ((outerRadius - innerRadius) \x \sin \alpha, \quad (outerRadius - innerRadius) \x \cos \alpha, \quad 0)\]
\[B' = ((outerRadius - innerRadius) \x \sin (\alpha + \alpha_{step}), \quad (outerRadius - innerRadius)  \x \cos (\alpha + \alpha_{step}), \quad 0)\]

\[p_0 =
\begin{pmatrix}
    B_x + (innerRadius \x \cos \beta \x \sin \alpha)\\
    B_y + (innerRadius \x \cos \beta \x \cos \alpha)\\
    B_z + (innerRadius \x \sin \beta)\\
\end{pmatrix}
\]
\[p_1 =
\begin{pmatrix}
    B'_x + (innerRadius \x \cos \beta \x \sin(\alpha + \alpha_{step}))\\
    B'_y + (innerRadius \x \cos \beta \x \cos(\alpha + \alpha_{step}))\\
    B'_z + (innerRadius \x \sin \beta)\\
\end{pmatrix}
\]
\[p_2 =
\begin{pmatrix}
    B_x + (innerRadius \x \cos(\beta + \beta_{step}) \x \sin \alpha)\\
    B_y + (innerRadius \x \cos(\beta + \beta_{step}) \x \cos \alpha)\\
    B_z + (innerRadius \x \sin(\beta + \beta_{step})) \\
\end{pmatrix}
\]
\[p_3 =
\begin{pmatrix}
    B'_x + (innerRadius \x \cos(\beta + \beta_{step}) \x \sin(\alpha + \alpha_{step}))\\
    B'_y + (innerRadius \x \cos(\beta + \beta_{step}) \x \cos(\alpha + \alpha_{step}))\\
    B'_z + (innerRadius \x \sin(\beta + \beta_{step}))\\
\end{pmatrix}
\]

Para cada rectângulo desenhado, o $\beta$ avança $\beta_{step}$ e para cada anel terminado $\alpha$ avança $\alpha_{step}$.

\section{Sistema Solar}

\section{Gestão de memória}

\section{Conclusões e Trabalho Futuro}

\end{document}

