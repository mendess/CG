\documentclass[a4paper]{article}

\usepackage[utf8]{inputenc}
\usepackage[portuguese]{babel}
\usepackage{graphicx}
\usepackage{a4wide}
\usepackage[pdftex,hidelinks]{hyperref}
\usepackage{float}
\usepackage{indentfirst}
\usepackage{subcaption}
\usepackage[cache=false]{minted}
\usepackage{amsmath}

\begin{document}

\title{Computação Gráfica\\ Primitivas gráficas}
\author{Bárbara Cardoso (a80453) \and Marcio Sousa (aXXXXX) \and Pedro Mendes (a79003)}
\date{\today}

\begin{titlepage}

    %título
    \thispagestyle{empty}
    \begin{center}
        \begin{minipage}{0.75\linewidth}
            \centering
            %engenharia logo
            \includegraphics[width=0.4\textwidth]{eng.jpeg}\par\vspace{1cm}
            \vspace{1.5cm}
            %títulos
            \href{https://www.uminho.pt/PT}{\scshape\LARGE Universidade do Minho} \par
            \vspace{1cm}
            \href{https://www.di.uminho.pt/}{\scshape\Large Departamento de Informática} \par
            \vspace{1.5cm}

            \maketitle
        \end{minipage}
    \end{center}

\end{titlepage}

\tableofcontents

\pagebreak

\section{Introdução}
Este trabalho foi realizado no âmbito da unidade curricular de Computação Gráfica, e tem como objectivo desenvolver um motor gráfico genérico para representar objectos a 3 dimensões.

O projecto está dividido em varias fazes, sendo esta a primeira onde são implementadas primitivas gráficas, como o plano, paralelepípedo, esfera e cone. Há também uma primeira implementação de dois modos de camera.

\section{Arquitectura do Projecto}
O projecto está dividido em dois executáveis, o \texttt{generator} e o \texttt{engine}, ao lado destes temos também um biblioteca comum aos dois, apropriadamente nomeada \texttt{common}.

\subsection{Common}

Aqui está apenas defindo um \texttt{Point.cpp/hpp} que representa um ponto no espaço 3D.

\subsection{Engine}

Este modulo é composto pelo \texttt{model.cpp/hpp} que representa cada modelo, ou seja, cada conjunto de pontos, disponibilizando também um método para o desenhar. E a \texttt{main.cpp} que carrega todos os ficheiros especificados no ficheiro \texttt{config.xml} e desenha-os no ecrã. Esta define também a lógica que controla a camera.

\subsection{Generator}

Este modulo é composto pelo \texttt{generator.cpp/hpp} que disponibiliza metodos para gerar os pontos necessários para desenhar as 4 primitivas geométricas e a \texttt{main.cpp} que interpreta os argumentos da linha de comandos para chamar estes métodos e produzir um ficheiro com estes.

\section{Primitivas}
\subsection{Plano}

\subsection{Paralelepípedo}

\subsection{Esfera}

\subsection{Cone}

\section{Camera}
A camera pode ser utilizada em dois modos, \textit{Explorer mode} e \textit{FPS mode}, que podem ser alternados durante a execução do engine.

\subsection{Explorer Mode}

Neste modo o movimento da camera pode ser descrito por uma superfície esférica na qual o utilizador se movimenta, enquanto mantém o seu olhar fixo na origem do referencial. O raio desta esfera pode ser aumentado para afastar o utilizador do centro do espaço 3d.

\subsubsection{Controlos \textit{(key bindings)}}

\begin{itemize}
    \item \texttt{H}: Mover no plano xz no sentido dos ponteiros do relógio
    \item \texttt{J}: Descer
    \item \texttt{K}: Subir
    \item \texttt{L}: Mover no plano xz no sentido contrario dos ponteiros do relógio
    \item \texttt{I}: Reduzir o raio da esfera
    \item \texttt{O}: Aumentar o raio da esfera
\end{itemize}

\subsection{FPS Mode}

Neste modo a posição da camera no espaço 3d é completamente livre, podendo o utilizador se mover em linha recta em qualquer direcção. Pode também alterar o ponto para o qual está a olhar na horizontal e na vertical.

\subsubsection{Controlos \textit{(key bindings)}}

\begin{itemize}
    \item \texttt{W}: Mover para a frente
    \item \texttt{A}: Andar para a esquerda
    \item \texttt{S}: Andar para trás
    \item \texttt{D}: Andar para a direita
    \item \texttt{H}: Olhar para a esquerda
    \item \texttt{J}: Olhar para baixo
    \item \texttt{K}: Olhar para cima
    \item \texttt{L}: Olhar para a direita
    \item \texttt{G}: Descer
    \item \texttt{Shift+G}: Subir
\end{itemize}

\subsection{Transição de modos de camera}

Para que a transição seja \textit{smooth} foi necessário converter de um sistema de coordenadas para o outro.

Quando a camera se encontra no modo \textit{Explorer} a sua posição é determinada pelo vector definido através dos ângulos $\alpha$ e $\beta$ e a distância à origem. E o ponto para onde está a olhar é a origem. Logo para efectuar a transição para \textit{FPS} foi necessário encontrar um vector que quando aplicado na posição da camera, olhe para a origem.

\begin{figure}[H]
    \[
        \alpha_f =
        \begin{cases}
            \alpha_e + \pi & \quad \text{se } \alpha_e < 0\\
            \alpha_e - \pi & \quad \text{caso contrario}
        \end{cases}
    \]
    \[
        \beta_f = -\beta_e
    \]
    \caption{Transição de \textit{Explorer} para \textit{FPS}.}
\end{figure}

Quando a camera se encontra no modo \textit{FPS} é necessário encontrar o ângulos $\alpha$ e $\beta$ e o raio que definem um vector correspondente as coordenadas (\texttt{(x,y,z)}) da camera no momento da transição.

\begin{figure}[H]
    \[
        \alpha_e =
        \begin{cases}
            \arctan(x / z)       & \quad \text{se } z < 0\\
            \arctan(x / z) + \pi & \quad \text{caso contrario}
        \end{cases}
    \]
    \[
        \beta_e = \arcsin\bigg(\frac{y}{\sqrt{x^2 + y^2 + z^2}}\bigg)\\
    \]
    \[
        radius = \sqrt{x^2 + y^2 + z^2}
    \]
    \caption{Transição de \textit{FPS} para \textit{Explorer}.}
\end{figure}

\subsection{Outros controlos \textit{(key bindings)}}

\begin{itemize}
    \item \texttt{+} Aumenta a escala de todos os modelos
    \item \texttt{-} Diminui a escala de todos os modelos
    \item \texttt{V} Alterna o modo da camera
    \item \texttt{Q} Termina o programa
\end{itemize}

\section{Conclusões e Trabalho Futuro}
Em suma, todas as primitivas são geradas correctamente e o motor gráfico é capaz de as representar genericamente apenas utilizado os pontos que as definem.

Futuramente será melhorada a camera para utilizar o rato no modo \textit{FPS} para determinar a direcção em que o utilizador está a olhar.

\end{document}
