\documentclass[a4paper]{article}

\usepackage[utf8]{inputenc}
\usepackage[portuguese]{babel}
\usepackage{graphicx}
\usepackage{a4wide}
\usepackage[pdftex,hidelinks]{hyperref}
\usepackage{float}
\usepackage{indentfirst}
\usepackage{subcaption}
\usepackage[cache=false]{minted}

\begin{document}

\title{Computação Gráfica\\ Primitivas gráficas}
\author{Bárbara Cardoso (a80453) \and Marcio Sousa (aXXXXX) \and Pedro Mendes (a79003)}
\date{\today}

\begin{titlepage}

    %título
    \thispagestyle{empty}
    \begin{center}
        \begin{minipage}{0.75\linewidth}
            \centering
            %engenharia logo
            \includegraphics[width=0.4\textwidth]{eng.jpeg}\par\vspace{1cm}
            \vspace{1.5cm}
            %títulos
            \href{https://www.uminho.pt/PT}{\scshape\LARGE Universidade do Minho} \par
            \vspace{1cm}
            \href{https://www.di.uminho.pt/}{\scshape\Large Departamento de Informática} \par
            \vspace{1.5cm}

            \maketitle
        \end{minipage}
    \end{center}

\end{titlepage}

\tableofcontents

\pagebreak

\section{Introdução}
Este trabalho foi realizado no âmbito da unidade curricular de Computação Gráfica, e tem como objectivo desenvolver um motor gráfico genérico para representar objectos a 3 dimensões.

O projecto está dividido em varias fazes, sendo esta a primeira onde são implementadas primitivas gráficas, como o plano, paralelepípedo, esfera e cone. Há também uma primeira implementação de dois modos de camera.

\section{Arquitectura do Projecto}
O projecto está dividido em dois executaveis, o \texttt{generator} e o \texttt{engine}, ao lado destes temos também um biblioteca comum aos dois.

\section{Primitivas}
\subsection{Plano}
\subsection{Paralelepípedo}
\subsection{Esfera}
\subsection{Cone}

\section{Camera}

\section{Controlos \textit{(key bindings)}}

\section{Conclusões e Trabalho Futuro}
Em suma, todas as primitivas são geradas correctamente e o motor gráfico é capaz de as representar genericamente.

Futuramente será melhorada a camera para utilizar o rato no modo \textit{FPS}.

\end{document}
