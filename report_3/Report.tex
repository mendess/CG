\documentclass[a4paper]{article}
\usepackage[utf8x]{inputenc}
\usepackage[portuguese]{babel}
\usepackage{graphicx}
\usepackage{a4wide}
\usepackage[pdftex,hidelinks]{hyperref}
\usepackage{float}
\usepackage{indentfirst}
\usepackage{subcaption}
\usepackage[cache=false]{minted}
\usepackage{amsmath}
\usepackage{listings}
\usepackage{color}
\usepackage{gensymb}

\definecolor{dkgreen}{rgb}{0,0.6,0}
\definecolor{gray}{rgb}{0.5,0.5,0.5}
\definecolor{mauve}{rgb}{0.58,0,0.82}

\lstset{frame=tb,
language=C++,
aboveskip=3mm,
belowskip=3mm,
showstringspaces=false,
columns=flexible,
basicstyle={\small\ttfamily},
numbers=none,
numberstyle=\tiny\color{gray},
keywordstyle=\color{blue},
commentstyle=\color{dkgreen},
stringstyle=\color{mauve},
breaklines=true,
breakatwhitespace=true,
tabsize=4
}

\newcommand{\x}{\times}

\begin{document}

\title{Computação Gráfica\\ Animações}
\author{Bárbara Cardoso (a80453) \and Márcio Sousa (a82400) \and Pedro Mendes (a79003)}
\date{\today}

\begin{titlepage}

    %título
    \thispagestyle{empty}
    \begin{center}
        \begin{minipage}{0.75\linewidth}
            \centering
            %engenharia logo
            \includegraphics[width=0.4\textwidth]{eng.jpeg}\par\vspace{1cm}
            \vspace{1.5cm}
            %títulos
            \href{https://www.uminho.pt/PT}{\scshape\LARGE Universidade do Minho} \par
            \vspace{1cm}
            \href{https://www.di.uminho.pt/}{\scshape\Large Departamento de Informática} \par
            \vspace{1.5cm}

            \maketitle
        \end{minipage}
    \end{center}

\end{titlepage}

\tableofcontents

\pagebreak

\section{Introdução}

% Este relatorio assume que foi feita a leitura dos dois anteriores

\section{Arquitetura do Projecto}

\subsection{Model}

Os modelos passaram a ser desenhados com \textit{Vertex Buffer Objects} (VBOs). Estes permitem reduzir o número de pedidos feitos à placa gráfica, aumentando assim, significativamente, a performance do programa.

Por outro lado, perderam a possibilidade de ser desenhados com todos os seus triângulos pintados de cores aleatórias, visto que não é possível especificar a cor de cada triângulo quando estes são todos enviados para o GPU simultaneamente.

Esta desvantagem irá, no entanto, ser mitigada na próxima fase com a adição de luz e texturas.

\subsection{Transformations}

Para que fosse possível animar os modelos, para cada transformação foi criada uma nova versão \textit{animada}: \texttt{TranslateAnimated}, \texttt{RotateAnimated}, \texttt{ScaleAnimated}.

Todas estas recebem o tempo que demoram a completar a animação. Quando uma animação termina recomeça imediatamente.

\subsubsection{TranslateAnimated}

Esta transformação recebe um conjunto de pontos que definem um caminho pelo qual o modelo deve viajar assim como o tempo que deve demorar a percorrer esse caminho.

Depois, fazendo uso do número de milissegundos que já passaram desde o início do programa, calcula, usando o algoritmo de \textit{CatmulRom}, o ponto no caminho em que o modelo deve estar para esse milissegundo.

\subsubsection{RotateAnimated}

A rotação animada, ao contrário da sua versão estática, recebe uma duração em vez de receber um ângulo, mantendo os outros parâmetros. Esta duração é mais tarde usada para determinar quanto tempo demora o objeto a efetuar uma rotação de 360\degree{} graus.

\subsubsection{ScaleAnimated}

A escala animada adiciona aos antigos rácios para $x, y, z$ um novo triplo de rácios, e para animar vai, simplesmente, ``progredindo'' de um rácio para outro durante o tempo que a animação demora. Sendo que, no início e no fim da animação se encontra no primeiro triplo.

\subsubsection{Group}

O método de desenho recebe agora o tempo que passou desde o início do programa, para que o possa passar às várias transformações, assegurando assim que todas usam o mesmo valor de tempo.

\subsection{\textit{Follow mode}}

Neste momento os models deslocam-se, o que torna mais complicado de os ver. Assim, para resolver este problema
foi implementado o \textit{Follow mode}. Neste o utilizador pode escolher um \textit{model}
para a câmara seguir. Sendo que, o comportamento da câmara neste modo é igual ao antigo
\textit{explorer mode} apenas com a diferença do centro, em que no modo antigo era na origem do referencial e agora passou a ser objeto.

Para conseguir este efeito, existiram duas grandes alterações fundamentais que tiveram de ser feitas no \textit{Group} e no
\textit{Câmara}.

\subsubsection{Group}

Para poder focar a câmara no objeto é necessário obter a sua posição no espaco 3D. Embora que os
\textit{models} em si não têm coordenadas próprias, ou seja, são transladados, rotacionados e escalados para as suas posições. Logo, é necessário que o \textit{Group} tenha a capacidade de calcular as coordenadas de cada modelo.

Para este fim, foi implementado um método que dado um índice vai multiplicando todas as matrizes de transformação até chegar ao modelo, e por fim usa a matriz final para extrair as coordenadas do modelo.

%TODO: Grafico para mostrar a selecao do modelo

\subsubsection{Câmara}

A câmara, por outro lado, sofreu uma grande re-estruturação do seu antigo modo \textit{Explorer}.
Foi então concluído que o \textit{Follow mode} era o caso geral deste, ou seja, o \textit{Explorer mode}
permitia ao utilizador focar-se na origem enquanto se deslocava na superfície de uma esfera (invisível), podendo-se alterar o raio desta esfera. Por outro lado, o \textit{Follow mode} permite
que ao utilizador fazer tudo isto, mas focando-se em qualquer ponto do espaço 3D. Desta forma, o \textit{Explorer mode} foi substituído pelo novo, \textit{Follow mode}.

\section{Generator}

Uma nova funcionalidade foi adicionada ao \texttt{generator}. Este é agora capaz de ler \textit{patch files} que definem superfícies Bezier e transforma-los nos ficheiros \texttt{.3d} que o \texttt{engine} está preparado para receber.

Um \textit{patch file} contém duas seções: \textit{patches} e pontos de controlo.

Os pontos de controlo contém triplos de coordenadas que irão ser usados pelos \textit{patches}, enquanto que os \textit{patches} contêm um \textit{patch} por linha.

\subsection{Patch}

Um \textit{patch} é um conjunto de 16 pontos que definem uma superfície de Bezier (cada linha contém, na verdade, 16 índices que referenciam os pontos de controlo da superfície).

%TODO: Explain Bezier

\section{Alterações dos ficheiros de input}\label{sec:estrutura-ficheiros}

O XML de input utilizado foi estendido para possibilitar a utilização das novas funcionalidades, no entanto, foi mantida a \textit{backwards compatibility} para o formato anterior. De forma a que modelos antigos continuem a ser validos.

\section{Key Bindings}

Novas \textit{key bindings} foram adicionadas:

\begin{itemize}
    \item \texttt{Shift+I} e \texttt{Shift+O}: Similares ao \textit{i} e
        \texttt{o} permitem diminuir e aumentar o raio do \textit{Follow mode}.
    \item \verb!{! e \verb!}!: Permite seleccionar o modelo a seguir.
    \item \texttt{,}: Liga/Desliga as rotas dos modelos em translação.
    \item \verb!(! e \verb!)!: Aumenta e Diminui a velocidade com que o tempo
        passa.
    \item \texttt{P}: Liga/Desliga pausa.
\end{itemize}

\section{Sistema Solar}

No Sistema Solar cada um dos astros tem a sua própria órbita e esta é animada, à excepção do sol obviamente.
As órbitas dos planetas são órbitas elípticas, tentando assim retratar o sistema solar real. As órbitas das luas de cada planeta são órbitas circulares.
Além destes astros, planetas e respetivas luas, foi adicionado também ao sistema solar uma cintura de asteróides que se situa entre Marte e Júpiter e um cometa com uma órbita elíptica muito alongada.

\section{Conclusões e Trabalho Futuro}

Em suma, todas as funcionalidades pedidas foram implementadas assim como alguns extras. No entanto, algumas optimizações não foram implementadas, como por exemplo, o uso de índices no VBOs. Tencionamos adicionar isto na proxima fase.

\end{document}
