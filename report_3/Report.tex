\documentclass[a4paper]{article}
\usepackage[utf8x]{inputenc}
\usepackage[portuguese]{babel}
\usepackage{graphicx}
\usepackage{a4wide}
\usepackage[pdftex,hidelinks]{hyperref}
\usepackage{float}
\usepackage{indentfirst}
\usepackage{subcaption}
\usepackage[cache=false]{minted}
\usepackage{amsmath}
\usepackage{listings}
\usepackage{color}

\definecolor{dkgreen}{rgb}{0,0.6,0}
\definecolor{gray}{rgb}{0.5,0.5,0.5}
\definecolor{mauve}{rgb}{0.58,0,0.82}

\lstset{frame=tb,
language=C++,
aboveskip=3mm,
belowskip=3mm,
showstringspaces=false,
columns=flexible,
basicstyle={\small\ttfamily},
numbers=none,
numberstyle=\tiny\color{gray},
keywordstyle=\color{blue},
commentstyle=\color{dkgreen},
stringstyle=\color{mauve},
breaklines=true,
breakatwhitespace=true,
tabsize=4
}

\newcommand{\x}{\times}

\begin{document}

\title{Computação Gráfica\\ Animações}
\author{Bárbara Cardoso (a80453) \and Márcio Sousa (a82400) \and Pedro Mendes (a79003)}
\date{\today}

\begin{titlepage}

    %título
    \thispagestyle{empty}
    \begin{center}
        \begin{minipage}{0.75\linewidth}
            \centering
            %engenharia logo
            \includegraphics[width=0.4\textwidth]{eng.jpeg}\par\vspace{1cm}
            \vspace{1.5cm}
            %títulos
            \href{https://www.uminho.pt/PT}{\scshape\LARGE Universidade do Minho} \par
            \vspace{1cm}
            \href{https://www.di.uminho.pt/}{\scshape\Large Departamento de Informática} \par
            \vspace{1.5cm}

            \maketitle
        \end{minipage}
    \end{center}

\end{titlepage}

\tableofcontents

\pagebreak

\section{Introdução}

% Este relatorio assume que foi feita a leitura dos dois anteriores

\section{Alterações dos ficheiros de input}\label{sec:estrutura-ficheiros}

\section{Arquitetura do Projecto}

\subsection{Model}
% VBO

\subsection{Transformations}
% Versoes animadas das transformacoes

\subsection{\textit{Follow mode}}
Porque agora os models se deslocam e mais dificil de os ver, para resolver este problema
foi implementado o \textit{Follow mode}. Neste o utilizador pode escolher um \textit{model}
para a camera seguir. Sendo que o comportamento da camera neste modo e igual ao antigo
\textit{explorer mode} apenas com o centro no objecto e nao na origem do referencial

Para conseguir este efeito duas grandes alteracoes tiveram de ser feitas ao \textit{Group} e ao
\textit{Camera}.

\subsubsection{Group}

Para poder focar a camera no objecto e necessario obter a sua posicao no espaco 3D. Mas os
\textit{models} em si nao tem coordenadas proprias, sao transladados, rotacionados e escalados para as suas posicoes. Logo e necessario que o \textit{Group} tenha a capacidade de calcular as coordenadas de cada modelo.

Para este fim, foi implementado um método que dado um índice vai multiplicando todas as matrizes de transformação ate chegar ao modelo, e por fim usa a matriz final para extrair as coordenadas do modelo.

\subsubsection{Camera}

A camera, por outro lado, sofreu uma grande re-estruturacao do antigo modo \textit{Explorer}.
Foi concluido que o \textit{Follow mode} era o caso geral deste, ou seja, o \textit{Explorer mode}
permitia ao utilizador focar-se na origem enquanto se deslocava na superficie de uma esfera (invisivel), podendo alterar o raio desta esfera. Por outro lado, o \textit{Follow mode} permite
que o utilizador faca tudo isto mas focando-se em qualquer ponto do espaco 3D. Desta forma o \textit{Explorer mode} foi substituido pelo novo, \textit{Follow mode}.

\section{Key Bindings}

% I/O (move in fast)
% { } (change follow target)
% ,  (show orbits)

\section{Sistema Solar}

% Topicos
% Todos os astros (tirando o sol) tem orbitas animadas
% Os planetas tem orbitas elipticas, para imitar os sistema solar real
% As luas dos planetas tem orbitas circulares
% Foi adicionada tambem uma cintura de asteroides entre Marte e Jupiter
% E foi adicionado um cometa que tem uma orbita muito alongada


\section{Conclusões e Trabalho Futuro}

\end{document}
